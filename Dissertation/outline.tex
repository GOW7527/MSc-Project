\documentclass[12pt]{article}
\begin{document}
\tableofcontents
\newpage
\section{Introduction}
\section{Stroboscopic measurement protocol}
\subsection{Loschmidt Echo and First Detection Amplitudes}
\subsection{Generating Function Approach}
\subsection{Spectral Dimension Approach}
\subsection{Mean Return Time}
\section{Tight Binding Model}
\subsection{Eigenstates}
\subsection{Energies}
\section{First Detected Return of a Quantum Walker}
\subsection{Contour Integration}
\subsection{Spectral Dimension}
\subsection{Mean Return Time}
\section{Anderson Model}
\subsection{Numerical Analysis}
Why we have to perform numerical analysis? and how is the numerical analysis performed.
Numerics are used because $\rho(E)$ does not exist or $E_k$ is not known analytically. There are no singularities in $\rho(E)$ its a smooth function. 
We simulate large systems and expect FSE effects to not take place, use picture()\\
\section{Probability Distribution of $F_k$}
Look at behaviour of $F_k$ for $W,\tau\gg1$ and show $1/k^2$ collapse of the PDF.
\section{Average behaviour of $F_k$}
Extend this average behaviour for different $W$ and $\tau$
\section{Ergodicity and Quantum Scarring and Finite Size Effects}
\section{Off-Diagonal Disorder}
\section{Outlook}
Higher Dimensional Disordered Systems, N particle disordered systems, localization features are distinct for dimension 1 and 2. First Passage for classical systems and see if there is an exponent doubling.
\end{document}



