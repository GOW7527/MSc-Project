\documentclass{article}
\usepackage[utf8]{inputenc}
\usepackage{braket}
\usepackage{amsmath}
\usepackage{amssymb}
\usepackage{physics}
\usepackage{graphicx}
\usepackage{esint}
\usepackage[margin=2cm]{geometry}
\usepackage{bbold}
\usepackage{biblatex}
\addbibresource{reference.bib}
\numberwithin{equation}{section}
\newcommand{\R}{\mathbb{R}}
\renewcommand{\H}{\mathcal{\hat{H}}} %\hat{\mathcal{H}} ?
\newcommand{\1}{\mathbb{1}}
\newcommand{\state}{\ket{\phi}}
\newcommand{\projection}{\bra{\phi}}
\newcommand{\annihilation}{c^{}}
\newcommand{\creation}{c^{\dagger}}
\newcommand{\zphi}{\hat{\varphi}(z)}
% % 
% \usepackage[inline]{showlabels} % print labels in margin
% \usepackage[unicode,
%   colorlinks = true,
%             linkcolor = blue,
%             urlcolor  = blue,
%             citecolor = blue,
%             anchorcolor = blue,
%     filecolor=blue]{hyperref}
% %
\title{First Detection Time of a Quantum Walker}
\author{G. Li}
\date{July 2023}
\begin{document}
\begin{figure}
    \centering
    \includegraphics[scale=0.2]{Logo.png}
\end{figure}
\maketitle
\newpage
\tableofcontents
\newpage
\section{Introduction}
The first passage time problem refers to finding the probability distribution of a stochastic process to reach a certain states for the first time. A typical example is determining the first time that a random walker starting at the origin reaches a particular lattice site. The study of first passage time is of particular importance in physical, biological and financial systems \cite{}.




A quantum walker behaves differently from a classical random walker, although both processes are probabilistic in nature, one of the key differences is the measurement process. In a classical context the act of measuring the position of a random walker does not have any effect on the behaviour of a random walker




, this is in contrast a very different behaviour to a quantum walker, in which the act of measurement 
\section{First Detection Time}































The dynamics of a quantum particle differs from that of a classical particle in many forms but the biggest difference arises in the measurement process. 
\\
\\
\\
\\
Justification of measurements at a a sampling rate $\tau$. 
\\\\
The time evolution of a particle is defined by the Schroedinger's equation, Eq.(\ref{MFPT1}).
\begin{equation}\label{MFPT1}
    i\partial_t\ket{\psi}=\H\ket{\psi}
\end{equation}
The solution to Eq.(\ref{MFPT1}) is given by Eq.(\ref{MFPT2}).
\begin{equation}\label{MFPT2}
\ket{\psi(t)}=e^{-i\H t}\ket{\psi_0}=\hat{U}(t)\ket{\psi_0}, \quad \text{where} \quad {\ket{\psi_0}}\quad \text{is the initial state}
\end{equation}
Consider performing the first measurement at time $t=\tau-\varepsilon=\tau^-$ with $\varepsilon\searrow0$, the probability of measuring the particle in a state $\state$ is 
\begin{equation}\label{MFPT3}
    P_1=\abs{\projection\ket{\psi(\tau^-)}}^2
\end{equation}
If the measurement is successful then the first detection time is $t_f=\tau$, if the measurement is not successful than $P_1=0$ at time $t=\tau+\varepsilon=\tau^+$, $\varepsilon\searrow0$, this is due to the collapse of the wave function. Assuming that the particle is not detected then the wavefunction at time $t=\tau^+$ is
\begin{equation}\label{MFPT4} %\label{eq:survival_wvf}
    \ket{\psi(\tau^+)}=N\biggl(\1-\state\bra{\phi} \biggr)\ket{\psi(\tau^-)}
\end{equation}
where the idea behind Eq.(\ref{MFPT4}) is that the particle is in the state of $\ket{\psi(\tau^-)}$ but now with zero probability that is in the state $\state$. $N$ is the normalization factor, given that prior to the measurement the probability of not being in the state $\state$ is $P_1$ then $N=1/\sqrt{1-P_1}$ such that $\braket{\psi(\tau^+)}{\psi(\tau^+)}=1$.
\\
For the second measurement the procedure is very similar to the previous one, the probability of finding the particle in the state $\state$ conditioned that the first measurement is not successful is given by
\begin{equation}\label{MFPT5}
    P_2=\projection \hat{U}(\tau)\ket{\psi(\tau^+)}=\projection\ket{\psi(2\tau^-)}.
\end{equation}
Substituting Eq.(\ref{MFPT4}) into Eq.(\ref{MFPT5}) one obtains
\begin{equation}\label{MFPT6}
    P_2=\frac{\abs{\projection\hat{U}(\tau)(1-\hat{D})\hat{U}(\tau)\ket{\psi_0}}^2}{1-P_1}, \quad \text{where}\quad \hat{D}=\state\projection \quad \text{is defined as the projection operator.}
\end{equation}
This process iterated $n$ times yields the probability $P_n$ that a measurement is successful given $n-1$ null measurements
\begin{equation}\label{MFPT7}
    P_n=\frac
    %numerator
    {\abs
    {\projection \left[\hat{U}(\tau)(\1-\hat{D})\right]^{n-1}\hat{U}(\tau)\ket{\psi_0}}}
    %Denominator
    {(1-P_1)(1-P_{n-1})}.
\end{equation}
One can define the first detection wave function as
\begin{equation}\label{MFPT8}
    \ket{\theta_n}=\hat{U}(\tau)\left[(\1-\hat{D})\hat{U}(\tau)\right]^{n-1}\ket{\psi_0}
\end{equation}
which allows Eq.(\ref{MFPT7}) to be written as
\begin{equation}\label{MFPT9}
    P_n=\frac
    {\ev{\hat{D}}{\theta_n}}
    {\prod_{j=1}^{n-1}(1-P_j)}.
\end{equation}
Now Eq.(\ref{MFPT9}) only tells the probability of a particle existing in the state $\projection$ after $n-1$ unsuccessful measurements, it does not give the statistics for first detection on the $n$ measurements denoted by $F_n$. Consider the thought experiment of measuring the particle to be in the state $\ket{\phi}$ on the first measurement, the probability of that happening is given by $P_1$. Assume that the particle was not measured on the first measurement which occurs with probability $1-P_1$, then the first detection can occur on the second measurement with probability $F_2=(1-P_1)P_2$. If the particle is again not measured to be in the desired state on the second measurement then the probability of being detected for the first time on the third measurement is given by $F_3=(1-P_1)(1-P_2)P_3$. In general the probability of detecting the particle to be in the state $\state$ on the $n$th measurement for the first time is given by 
\begin{equation}\label{MFPT10}
    F_n=(1-P_1)(1-P_2)...(1-P_{n-1})P_n.
\end{equation}
substituting Eq.(\ref{MFPT9}) into Eq.(\ref{MFPT10}) 
\begin{equation}\label{MFPT11}
    F_n=\ev{\hat{D}}{\theta_n}.
\end{equation}
Another useful quantity worth mentioning is the survival probability $S_N$, i.e. the probability that the particle is not detected in the first $N$ attempts
\begin{equation}\label{MFPT12}
    S_N=1-\sum_{n=1}^NF_n,
\end{equation}
the process is said to be recurrent if $\lim_{N\to\infty}S_N=0$.


\subsection{Loschmidt Echo and First Detection Amplitudes}
Consider the first detection amplitude given by $\varphi_n=\projection\ket{\theta_n}$, such that $F_n=\abs{\varphi_n}^2$. Using Eq.(\ref{MFPT8}) and $\varphi_1=\projection\ket{\psi_0}$ one can show by induction that $\varphi_n$ can be computed by Eq.(\ref{MFPT13}), also known as the quantum renewal equations\cite{barkai}.
\begin{equation}\label{MFPT13}
    \varphi_n=\projection\hat{U}(n\tau)\ket{\psi_0}-\sum_{j=1}^{n-1}\varphi_j\projection\hat{U}((n-j)\tau)\state.   
\end{equation}
Eq.(\ref{MFPT13}) is of particular importance because it yields $\varphi_n$ in terms of a propagation free of measurement $\projection\hat{U}(n\tau)\ket{\psi_0}$; letting the initial state $\ket{\psi_0}$ to be the target state $\state$, then $\varphi_n$ can be determined by what is called the Loschmidt amplitude $\mathcal{L}$.
\\
The Loschmidt Echo, $\abs{\mathcal{L}(t)}^2$ is defined as the probability a particle being in the same state as the initial state $\ket{\psi_0}$ after some time $t$:
\begin{equation}\label{MFPT14}
    \abs{\mathcal{L}(t)}^2=\abs{\bra{\psi_0}\hat{U}(t)\ket{\psi_0}}^2.
\end{equation}
Eq.(\ref{MFPT13}) in terms of Loschmidt echos is given by Eq.(\ref{MFPT15}),
\begin{equation}\label{MFPT15}
    \varphi_n=\mathcal{L}_n-\sum_{j=1}^{n-1}\varphi_j\mathcal{L}_{n-j}, \quad \text{where} \quad \mathcal{L}_n=\mathcal{L}(n\tau)
\end{equation}
\subsection{Generating Function}
Consider the $Z$ transform of $\varphi_n$ defined by
\begin{equation}\label{MFPT16}
    \zphi=\mathcal{Z}[\varphi_n]=\sum_{n=1}^\infty z^{n}\varphi_n
\end{equation}
substituting Eq.(\ref{MFPT15}) into Eq.(\ref{MFPT16}) one obtains 
\begin{align}
\zphi&=\sum_{n=1}^\infty z^{n}\mathcal{L}_n-\sum_{n=1}^{\infty}z^{n}\sum_{j=1}^{n-1}\varphi_j\mathcal{L}_{n-j}
\\
&=\sum_{n=1}^\infty z^{n}\mathcal{L}_n-\sum_{n=1}^{\infty}z^{n}(\varphi * \mathcal{L})_n
\\
&=I(z)-\zphi I(z), \quad \text{where} \quad I(z)=\mathcal{Z}[\mathcal{L}_n] \label{MFPT19}
\end{align}
re-arranging one obtains the generating function for $\varphi_n$:
\begin{equation}\label{MFPT20}
    \zphi=\frac{I(z)}{1+I(z)}
\end{equation}
On line (\ref{MFPT19}) the convolution property of the $Z$ transform\footnote{ $\mathcal{Z} [(x * y)_n]=\mathcal{Z}[x_n]\mathcal{Z}[y_n]$, \text{the symbol $*$ denotes a convolution: $(x*y)_n=\sum_k x_k y_{n-k}$}} was used.
\\
To obtain $\varphi_n$ from $\zphi$ one can simply use the inversion formulas of the $Z$ transform
\begin{equation}
    \varphi_n=\frac{1}{n!}\dv[n]{z}\zphi\eval_{z=0}
\end{equation}
or by the Cauchy Residue theorem
\begin{equation}
    \varphi_n=\frac{1}{2\pi i}\oint_C \zphi z^{-n-1} \dd{z}
\end{equation}
where $C$ is a counterclockwise path that contains the origin and is entirely within the radius of convergence of $\zphi$.


\section{Quantum Random Walk}
A quantum random walk can be described by using a Hamiltonian $\H$ of the form given by Eq.(\ref{QRW1}), where $\{\gamma\}$ are the hopping rate between lattice sites and $\{\varepsilon\}$ are the onsite potential energies.
\begin{equation}\label{QRW1}
    \H=\sum_{j}\gamma^{}_j(\creation_{j+1}\annihilation_j+\creation_{j}\annihilation_{j+1})+\varepsilon^{}_j\creation_j\annihilation_j,
\end{equation}
where $\creation,\annihilation$ are the creation and annihilation operators respectively.
A matrix representation of $\H$ can be constructed in the occupancy number basis of one particle $\ket{n}=\creation_n\ket{\nu}$, where $\ket{\nu}$ denotes the vacuum state. The matrix elements of $\H$ are given by
\begin{align}\label{QRW2}
    \H_{mn}&=\bra{m}\H\ket{n}   
    \\
    &=\sum_j \gamma_j\bra{m}\creation_{j}\annihilation_{j+1}\ket{n} +\gamma_j\bra{m}\creation_{j+1}\annihilation_{j}\ket{n} + \varepsilon^{}_j\bra{m}\creation_{j}\annihilation_{j}\ket{n} 
    \\
    &= \sum_j \gamma_j\bra{m}\creation_{j}\delta^{}_{j+1,n}\ket{\nu}
    +\gamma_j\bra{m}\creation_{j+1}\delta^{}_{j,n}\ket{\nu} +
    \varepsilon^{}_j\bra{m}\creation_{j}\delta^{}_{j,n}\ket{\nu} 
    \\
    &= \gamma_{n-1}\bra{m}\ket{n-1}+\gamma_{n+1}\bra{m}\ket{n+1}+\varepsilon_n\bra{m}\ket{n}
    \\
    &=\gamma_{n-1}\delta_{m,n-1}+\gamma_{n+1}\delta_{m,n+1}+\varepsilon_n\delta_{m,n}, \label{QRW3}
\end{align}
it is far more informative to look expression (\ref{QRW3}) in matrix form:
\begin{equation}
    \mqty(
    \gamma_1 & \varepsilon_1 & \gamma_1 & 0 & 0 &\cdots 
    \\
    0 & \gamma_2 & \varepsilon_2 & \gamma_2 & 0&\ddots
    \\
    \vdots & \ddots & \ddots & \ddots  & \ddots &\ddots
    )
\end{equation}
\section{Tight Binding Model}
The simplest model of the quantum random walk is given by Eq.(\ref{QRW1}) where $\varepsilon_i=0$ and $\gamma_i=1$, this model simply represents a particle jumping only to its neighbouring lattice site with equal probability. Assume that there are $L$ lattice sites and the boundaries are periodic, then by Bloch Theorem \cite{} the eigenstates $\ket{\phi_k}$ are
\begin{equation}
    \ket{\phi_k}=\sum_{n=1}^L \frac{e^{2\pi i kn/L}}{\sqrt{L}}\ket{n}
\end{equation}
The Hamiltonian in the $\ket{\phi_k}$ basis is
\begin{align}
    \H_{jk}=\bra{\phi_j}\H\ket{\phi_k}&=\frac{1}{L}\left(
    \sum_{n,m=1}^Le^{ 2\pi i (kn-jm)/L} \braket{m}{n-1}+\sum_{n,m=1}^L e^{2\pi i(kn-jm)/L}\braket{m}{n+1}
    \right)
    \\
    &=\frac{1}{L}\left(
    \sum_{n,m=1}^Le^{2\pi i (kn-jm)/L} \delta_{m,n-1}+\sum_{n,m=1}^L e^{2\pi i(kn-jm)/L} \delta_{m,n+1}
    \right)
    \\
    &=\frac{1}{L}\left(
    \sum_{n=1}^L e^{2\pi i(kn-jn+j)/L} + \sum_{n=1}^L e^{2\pi i (kn-jn-j)/L}
    \right)
    \\
    &=\left(e^{2\pi i j/L}+e^{-2\pi ij/L}\right)\frac{1}{L}\sum_{n=1}^L e^{2\pi i n (k-j)/L}
    \\
    &=2 \cos (\frac{2 \pi j}{L}) \delta_{j,k}.
\end{align}
Hence, the energy spectrum of the tight binding model is $E_k=2\cos(2\pi k/L)$.
\subsection{Loschmidt Amplitude}
Let the target state and initial state be the origin $\ket{1}$ which equals to $\ket{1}=\frac{1}{\sqrt{L}}\sum_{k=1}^{L}e^{2\pi i k/L}\ket{\phi_k}$ in the eigenstates of $\hat{H}$ basis, the Loschmidt amplitude is then given by 
\begin{align}
    \mathcal{L}(t)&=\expval{e^{-i \H t}}{1},
    \\
    &=\frac{1}{L}\sum_{k=1}^{L}e^{-2i\cos(2\pi k/L)}. \label{TBT8}
\end{align}
In the continuum limit the sum in Eq.(\ref{TBT8}) turns into an integral
\begin{align}
    \mathcal{L}(t)&=\lim_{L\to\infty}\frac{1}{L}\sum_{k=1}^{L}e^{-2i\cos(2\pi k/L)},
    \\
    &=\int_0^1 e^{2i\cos(2\pi x)} \dd x,
    \\
    &=J_0(2t)
\end{align}
where $J_0(...)$ is the zeroth Bessel function of the first kind.






































\section{Email}
I have only a small comment on the analytics of the FDT provided the Loschmidt amplitude indeed becomes constant.

Then, if $L(\tau) = \lambda$, one finds that

$$\hat{U}(z) = \sum_{n \geq 1} \lambda z^n = \lambda \frac{z}{1-z}$$

and hence

$$\hat{\phi}(z) = \frac{\hat{U}(z)}{1 + \hat{U}(z)} = \frac{\lambda z (1-z)^{-1}}{ 1 + \lambda z (1-z)^{-1} } = \frac{\lambda z }{1 + (\lambda - 1) z}$$

Expanding in z, one finds that

$$\phi_n = \lambda(1-\lambda)^{n-1}$$

This is a geometric random distribution; if $\lambda$ is your success rate, then $\phi_n$ is the probability that after n-1 unsuccessful trials, the n-th is successful. This makes sense since $\lambda$ is the detection amplitude. Hence,

$$F_n = C \exp( - n / \ell )$$

with $C = [\lambda/(1-\lambda)]^2$, and characteristic

$$\ell^{-1} = 2 \ln [1 / (1 - \lambda)]$$

so we just need to find $\lambda$. Although this is a calculation for the large $\tau$ limit, it would be possible that it also explains well the large-n regime in $F_n$ which I remember seemed exponentially suppressed to me (?).

On a different note, in the large $W$ limit, we should be able to (like in the paper) neglect off-diagonal terms and assume that the Loschmidt amplitude is approximated by

$$<0| e^{-i \epsilon_k \tau} | 0> = \int_{-W}^W \int{w} e^{i w \tau} = 2 \frac{\sin(W \tau)}{\tau}$$

and then the Loschmidt echo should go like

$$4 \sin^2(W \tau) \tau^{-2}$$

It doesn’t really work with the plots you showed me though, as there the terms go against a constant for large $\tau.$ I’d be curious though if the oscillations are somehow well captured by the $\sin^2(W \tau)$ term? Or does it not work at all? Is there something like a constant offset?
\printbibliography  
\end{document}
